\documentclass[letterpaper]{article}

\usepackage[T1]{fontenc}
\usepackage{ae,aecompl}

\usepackage[utf8]{inputenc}


\usepackage{lmodern}

% Entity-relationsip package
\usepackage{tikz-er2}
% Include tikz library for more control over positioning
\usetikzlibrary{positioning}
% Styling for entities, attributes, and relationships
\tikzstyle{every entity} = [draw=blue!50!black!100, fill=blue!20]
\tikzstyle{every attribute} = [draw=yellow!50!black!100, fill=yellow!20]
\tikzstyle{every relationship} = [draw=orange!50!black!100, fill=orange!40]
\tikzstyle{every superclas} = [top color=white, bottom color=red!20, 
                         draw=red!50!black!100]

\usepackage{parskip}


\pagestyle{empty}


\begin{document}


\section{Introducción}

Tras el fin del confinamiento, y la vuelta a la normalidad, usted decide ir al LolaPelusa (un mega recital que reúne a múltiples artistas), de este se conoce el nombre y número de la edición (siendo distintos para cada edición), el cual se realiza en un recinto privado, del cual se conoce el nombre, Rut del recinto, capacidad máxima de asistentes, y un informe que contiene los nombres y los equipos de todos los guardias disponibles. Además, se sabe que, para que el LolaPelusa se lleve a cabo, se deben presentar como máximo 50 shows en un día y como mínimo 25 shows, además, LolaPelusa tiene una regla de “autenticación”, la cual dice “los shows presentados solo deben ser realizados, como máximo, en 5 escenarios distintos, independientes del nivel de espectadores”.

Por otro lado, del show se conoce el número propio que indica en qué orden se presentan, la hora de inicio, hora de fin y duración del show. Cabe destacar que los shows, solo pueden ser realizados por un artista a la vez, y claramente, como no existe show sin artista, de estos últimos se debe registrar su nombre real, nombre artístico, rut, y su cuenta corriente (para realizar el pago, la cual se compone de número y nombre banco). Finalmente, se sabe que, cada vez que un artista realice un show, se generará un voucher con un ID y monto, y que los artistas pueden votar por otro (o ninguno) como el mejor artista del evento, por lo cual debe registrarse el folio con el que votó cada uno.

Se pide que realice el MER, indicando claramente todos sus elementos.

\section{Desarrollo (Pauta)}


\newpage

\section{MER (Pauta)}

\end{document}
