\documentclass[letterpaper]{article}

\usepackage[T1]{fontenc}
\usepackage{ae,aecompl}

\usepackage[utf8]{inputenc}


\usepackage{lmodern}

% Entity-relationsip package
\usepackage{tikz-er2}
% Include tikz library for more control over positioning
\usetikzlibrary{positioning}
% Styling for entities, attributes, and relationships
\tikzstyle{every entity} = [draw=blue!50!black!100, fill=blue!20]
\tikzstyle{every attribute} = [draw=yellow!50!black!100, fill=yellow!20]
\tikzstyle{every relationship} = [draw=orange!50!black!100, fill=orange!40]
\tikzstyle{every superclas} = [top color=white, bottom color=red!20, 
                         draw=red!50!black!100]

\usepackage{parskip}


\pagestyle{empty}


\begin{document}
\section{Pregunta 1 - MERE}

Diseñe el MERE para el siguiente requerimiento:

El Emperador de una galaxia muy lejana necesita organizar sus misiones y
recursos que utiliza en ellas. Debido a esto, le solicita implementar una base de
datos que contenga toda la información relevante respecto a los temas críticos
en las misiones.
Primero, se sabe que las misiones poseen un código específico y un nombre.
Estas suelen ser de dos tipos: de conquista, que necesitan almacenar la
urgencia, y de destrucción, en las que se utilizará la estrella de la muerte para
completarla. Sobre la estrella de la muerte, necesitamos almacenar el nombre,
código único, el tiempo de disparo promedio, así como también la fecha y hora
exacta en que inicia y finaliza la operación de disparo. Adicionalmente, esta
puede terminar en tres estados: completada, en construcción, o inoperable. En
caso de estar completada, se debe almacenar la cantidad de gente que trabajó
en ella y la duración de esta construcción, además de la galaxia en la que
estará ubicada. Si está en construcción, se requiere la fecha de inicio del
proyecto y los recursos utilizados, tanto personas como dinero invertido.
Finalmente, en caso de estar inoperable, es necesario almacenar el nombre del
emperador y la fecha exacta en que se declaró como tal.
Para la misión de conquista se requiere el uso de las flotas disponibles que
poseen un número que las identifica y una potencia asociada. Dichas flotas
están compuestas por distintos tipos de naves: de soporte, destructores
estelares y cazas estelares. Es necesario almacenar el tamaño de las naves de
soporte, la capacidad de los destructores y el modelo del caza estelar. Cabe
decir, que los destructores alojan muchas naves cazas estelares, las cuales son
conducidas por un piloto. Un stormtrooper es un soldado perteneciente al
Imperio, el cual puede desarrollar variadas funciones: como ya se dijo puede
ser piloto y además, ingeniero o soldado de asalto. Sobre los stormtrooper
necesitamos saber la unidad a la que pertenece, el rango y su Id. Además,
debemos saber a la flota a la que pertenece.
Finalmente, cada piloto debe proteger a la estrella de la muerte de cualquier
amenaza, por lo que siempre que exista una batalla se debe registrar la
cantidad de naves que destruyó el piloto para defender a la estrella de la
muerte.

\section{Pregunta 2 - MR}

Derive el diagrama del esquema del Modelo Relacional MR a partir del siguiente MERE:

\section{Pregunta 3 - SQL}

Dado el MERE y MR de la base de datos LIBRERIAFB de MySQL, escriba
consultas de recuperación y/o actualización en SQL:

P3. a) Mostrar la cantidad total comprada de cada ítem, en orden
descendente.

P3. b) Mostrar la cantidad total comprada de los siete ítems más
adquiridos en la base de datos, indicando si es libro o estante.


P3. c) Con el fin de poder visualizar fácilmente los datos, se pide agregar
los siguientes atributos a la tabla ADQUIERE: nombre de los clientes,
nombre del vendedor y nombre del libro adquirido en la venta.
Adicionalmente, inserte los datos respectivos a estos nuevo atributos.


P3. d) Se solicita mostrar un listado de las transacciones realizadas por la
empresa, ya sean boletas u órdenes de compras, junto a la cantidad de
productos que fueron gestionados en cada una de ellas. Sin embargo,
estas deben ser únicamente transacciones cuya utilidad del producto
supera los $5.000



\section{Pregunta 4 - Normalización}
\end{document}
