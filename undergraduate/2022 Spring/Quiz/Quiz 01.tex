\documentclass[letterpaper]{article}

\usepackage[T1]{fontenc}
\usepackage{ae,aecompl}

\usepackage[utf8]{inputenc}


\usepackage{lmodern}

% Entity-relationsip package
\usepackage{tikz-er2}
% Include tikz library for more control over positioning
\usetikzlibrary{positioning}
% Styling for entities, attributes, and relationships
\tikzstyle{every entity} = [draw=blue!50!black!100, fill=blue!20]
\tikzstyle{every attribute} = [draw=yellow!50!black!100, fill=yellow!20]
\tikzstyle{every relationship} = [draw=orange!50!black!100, fill=orange!40]
\tikzstyle{every superclas} = [top color=white, bottom color=red!20, 
                         draw=red!50!black!100]

\usepackage{parskip}


\pagestyle{empty}


\begin{document}


\section{Introducción}

Debido a toda la situación de la pandemia, se generó un incremento en el consumo de servicios de streaming, especialmente en servicios de video. Es por esto que se requiere de su asesoría para modelar el Modelo Entidad Relación (MER) de una base de datos con los elementos más esenciales para las distintas compañías de streaming. Se sabe que las distintas compañías poseen un nombre único, fecha de fundación, compuesta por día, mes y año, ubicación de sus instalaciones y un CEO. Las compañías son las encargadas de distribuir contenido, el cual en ocasiones se puede encontrar en distintas plataformas. Cada contenido tiene su número identificador, título, duración, año de lanzamiento, tipo de contenido (película, serie, documental, etc.) y clasificación (PG13, R, etc.).

Las mismas compañías ofrecen distintos planes de suscripción a sus servicios de streaming, los cuales tienen su número identificador, tipo de plan, precio y la característica específica de cada uno.

Finalmente, es normal que diversas figuras del espectáculo, en específico actores, firmen contratos de exclusividad en ciertas compañías (ej: Adam Sandler con Netflix), los cuales al mismo tiempo participan en proyectos distribuidos por las compañías. De estos actores se conoce su DNI, su nombre, fecha de nacimiento, nacionalidad (de la cual se sabe podrían tener más de una), edad y género.

Para lo anterior, se solicita que escriba las notaciones completas de las cardinalidades del MER.

\section{Desarrollo (Pauta)}



\section{MER (Pauta)}

  
  
\end{document}
